\documentclass[a4paper, 11pt]{article}
\usepackage[slovene]{babel}
\usepackage[utf8]{inputenc}
\usepackage[T1]{fontenc}
\usepackage{amsfonts,amsmath,amssymb}
\usepackage{amsthm}
\usepackage{amsmath}
\usepackage{amssymb}

\newtheorem{Izrek}{Izrek}

\begin{document}

\thispagestyle{empty}
\begin{center}
\begin{minipage}{0.75\linewidth}
    \centering
    {\Large Univerza v Ljubljani \\ Fakulteta za matematiko in fiziko}
    \\
    \vspace{3cm}

    {\uppercase{\Large \textbf{The power of adjacent choices}}} \\ Finančni praktikum \\
    \vspace{3cm}

    {\Large Maša Orelj, Justin Raišp\par}
    \vspace{9cm}

    {\Large Ljubljana, 2022}
\end{minipage}
\end{center}

\newpage
 
\section{Navodilo}

Imamo $n$ žog in $n$ košev $b_1, \dots , b_n$, ki so prazni. Koši so postavljeni v krogu: koš $b_i$ ima soseda 
$b_{i-1}$ in $b_{i+1}$, kjer zaradi krožnosti velja, da sta $b_1$ in $b_n$ soseda.
Opazujemo naslednji naključen proces: Za vsako žogo naključno izberemo koš $b_i$, pogledamo še oba soseda in damo žogo 
v koš z najmanjšim številom žog v tistem trenutku izmed teh treh. Zanima nas število žog v košu z največ žogami na koncu procesa. 
Ta proces potem razširimo na več načinov:
\begin{itemize}
    \item za sosede štejemo koše, ki so na razdaljah največ $2, 3, \dots$,
    \item za $n$ košev vzamemo $2n, 3n, 4n, \dots$ žog,
    \item iščemo koš z najmanjšim številom žog.
\end{itemize}  
Opazujemo lahko tudi dvodimenzionalno mrežo košev s topologijo torusa. Torej imamo npr. $n^2$ košev $b_{i,j}$, kjer sta 
$i, j \in [n]$, kjer za soseda $b_{i,j}$ in $b_{k,l}$ velja $|i - k| + |j - l| = 1$. Soseda sta tudi
$b_{1,i}$ in $b_{n,i}$  ter $b_{i,1}$ in $b_{i,n}$. Podobno lahko gledamo tudi tridimenzionalno verzijo.

\end{document}