\documentclass[a4paper, 11pt]{article}
\usepackage[slovene]{babel}
\usepackage[utf8]{inputenc}
\usepackage[T1]{fontenc}
\usepackage{amsfonts,amsmath,amssymb}
\usepackage{amsthm}
\usepackage{amsmath}
\usepackage{amssymb}


\setlength\parindent{0pt}

\begin{document}

\thispagestyle{empty}
\begin{center}
\begin{minipage}{0.75\linewidth}
    \centering
    {\Large Univerza v Ljubljani \\ Fakulteta za matematiko in fiziko}
    \\
    \vspace{7cm}

    {\uppercase{\Large \textbf{The power of adjacent choices}}} \\ Finančni praktikum \\
    \vspace{3cm}

    Avtorja:\\
    {\Large Justin Raišp, Maša Orelj\par}
    \vspace{7cm}

    {\Large Ljubljana, 2022}
\end{minipage}
\end{center}

\newpage

\
\section{Navodilo}

Imamo $n$ žog in $n$ košev $b_1, \dots , b_n$, ki so prazni. Koši so postavljeni v krogu: koš $b_i$ ima soseda 
$b_{i-1}$ in $b_{i+1}$, kjer zaradi krožnosti velja, da sta $b_1$ in $b_n$ soseda.
Opazujemo naslednji naključen proces: Za vsako žogo naključno izberemo koš $b_i$, pogledamo še oba soseda in damo žogo 
v koš z najmanjšim številom žog v tistem trenutku izmed teh treh. Zanima nas število žog v košu z največ žogami na koncu procesa. 
Ta proces potem razširimo na več načinov:
\begin{itemize}
    \item za sosede štejemo koše, ki so na razdaljah največ $2, 3, \dots$,
    \item za $n$ košev vzamemo $2n, 3n, 4n, \dots$ žog,
    \item iščemo koš z najmanjšim številom žog.
\end{itemize}  
Opazujemo lahko tudi dvodimenzionalno mrežo košev s topologijo torusa. Torej imamo npr. $n^2$ košev $b_{i,j}$, kjer sta 
$i, j \in [n]$, kjer za soseda $b_{i,j}$ in $b_{k,l}$ velja $|i - k| + |j - l| = 1$. Soseda sta tudi
$b_{1,i}$ in $b_{n,i}$  ter $b_{i,1}$ in $b_{i,n}$. Podobno lahko gledamo tudi tridimenzionalno verzijo.



\section{Program za reševanje problema}
Za reševanje opisanega problema sva uporabila programski jezik \emph{Python}. Začela sva s pisanjem ustreznega programa za reševanje
naloge v eni dimenziji, pri katerem je možno spreminjanje števila sosedov(razdalje), koločine žog in količine košev. Pri tem sva privzela, da v
primeru, ko je košev z najmanjšim številom žog med pregledanimi koši več, položimo žogo v koš z najmanjšim indeksom. 

\begin{verbatim}
    #ISKANJE SOSEDOV
    def najdi_sosede_1d(kosi,izbrana_kosarica, razdalja=1):
    sosedi = {}
    st_kosev = len(kosi)
    for j in range((-razdalja),(razdalja+1)):
        najdi indeks soseda v listu s pomocjo modula, da velja kroznost
        v slovar sosedov dodaj njegov indeks kot kljuc in njegovo stevilo zog kot vrednost
    izmed sosedov poisi tistega z minimalno vrednost
    return sosedi, minimalna_vrednost
\end{verbatim}

\pagebreak
\begin{verbatim}
    #ISKANJE MAKSIMALNEGA STEVILA ZOG
    def maksimalno_stevilo_zogic(st_zogic, st_kosev, razdalja=1):
    zacetek merjenja casa
    kosi = [0]*st_kosev
    for i in range(st_zogic):
        nakljucno izberemo kosarico
        najdi_sosede_1d(kosi, izbrana_kosarica, razdalja) #s pomozno funkcijo doloci sosede in minimalno vrednost
        poisci vse kandidate, ki so sosedi in imajo minimalno vrednost
        izmed kandidatov izberi tistega z najnizjim indeksom
        kosi[minimalni_sosed] += 1 
    poisci maksimalno stevilo zog v kosu
    prestej stevilo kosev z maksimalnim stevilom zog
    izracunaj delez kosev z maksimalnim stevilom zog
    konec merjenja casa
    izracunaj casovno zahtevnost algoritma
    return maksimalno_stevilo_zog, casovna_zahtevnost, delez_kosev


\end{verbatim}

Funkcija $maksimalno\_stevilo\_zogic$ torej sprejme izbrano število žog, število košar in razdalja, vrne pa vrednost maksimalnega število žog v košari,
potreben čas za izvedbo funkcije in delež košar z maksimalnim številom žog.
\bigbreak
Program sva prilagodila tudi na dvodimenzionalno mrežo košev, ki deluje na podoben način.



\section{Analiza rezultatov}

\end{document}