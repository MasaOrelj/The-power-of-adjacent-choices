\documentclass[a4paper, 11pt]{article}
\usepackage[slovene]{babel}
\usepackage[utf8]{inputenc}
\usepackage[T1]{fontenc}
\usepackage{amsfonts,amsmath,amssymb}
\usepackage{amsthm}
\usepackage{amsmath}
\usepackage{amssymb}


\newtheorem{Izrek}{Izrek}

\begin{document}


\thispagestyle{empty}
\begin{center}
\begin{minipage}{0.75\linewidth}
    \centering
    {\Large Univerza v Ljubljani \\ Fakulteta za matematiko in fiziko}
    \\
    \vspace{7cm}

    {\uppercase{\Large \textbf{The power of adjacent choices}}} \\ Finančni praktikum \\
    \vspace{3cm}

    Avtorja:\\
    {\Large Justin Raišp, Maša Orelj\par}
    \vspace{7cm}

    {\Large Ljubljana, 2022}
\end{minipage}
\end{center}

\newpage
 
\section{Navodilo}

Imamo $n$ žog in $n$ košev $b_1, \dots , b_n$, ki so prazni. Koši so postavljeni v krogu: koš $b_i$ ima soseda 
$b_{i-1}$ in $b_{i+1}$, kjer zaradi krožnosti velja, da sta $b_1$ in $b_n$ soseda.
Opazujemo naslednji naključen proces: Za vsako žogo naključno izberemo koš $b_i$, pogledamo še oba soseda in damo žogo 
v koš z najmanjšim številom žog v tistem trenutku izmed teh treh. Zanima nas število žog v košu z največ žogami na koncu procesa. 
Ta proces potem razširimo na več načinov:
\begin{itemize}
    \item za sosede štejemo koše, ki so na razdaljah največ $2, 3, \dots$,
    \item za $n$ košev vzamemo $2n, 3n, 4n, \dots$ žog,
    \item iščemo koš z najmanjšim številom žog.
\end{itemize}  
Opazujemo lahko tudi dvodimenzionalno mrežo košev s topologijo torusa. Torej imamo npr. $n^2$ košev $b_{i,j}$, kjer sta 
$i, j \in [n]$, kjer za soseda $b_{i,j}$ in $b_{k,l}$ velja $|i - k| + |j - l| = 1$. Soseda sta tudi
$b_{1,i}$ in $b_{n,i}$  ter $b_{i,1}$ in $b_{i,n}$. Podobno lahko gledamo tudi tridimenzionalno verzijo.



\section{Načrt dela}
Za reševanje opisanega problema bova uporabila programski jezik \emph{Python}. Najprej bova napisala program, s katerim bova
lahko preizkušala vpliv spreminjanja števila sosedov in količine žog na poskus. Pri tem bova privzela, da v primeru, ko je  
košev z najmanjšim številom žog med pregledanimi koši več, položimo žogo v koš z najmanjšim indeksom. Za začetek sva
pripravila prvo različico programa, ki je namenjena najbolj osnovnemu primeru poskusa:
$$ \text{število košev} = n,\  \text{število žog} = n, \ \text{število levih oz.\ desnih sosedov} = 1 .$$

\pagebreak

\begin{verbatim}
#OSNOVNI PROGRAM
def najdi_sosede_1d(kosi,izbrana_kosarica, razdalja=1):
    sosedi = {}
    st_kosev = len(kosi)
    for j in range((-razdalja),(razdalja+1)):
        indeks_soseda = (izbrana_kosarica + j) % st_kosev #najde indeks soseda v listu s pomocjo modula, da velja kroznost
        sosedi[indeks_soseda] = kosi[indeks_soseda] #v slovar sosedov dodamo njegov indeks kot kljuc in njegovo stevilo zog kot vrednost
    min_vrednost = min(sosedi.values()) #izmed sosedov poiscemo minimalno vrednost
    return sosedi, min_vrednost

def maksimalno_stevilo_zogic(st_zogic, st_kosev, razdalja=1):
    zacetek = time.time()
    kosi = [0]*st_kosev
    for i in range(st_zogic):
        izbrana_kosarica = random.randrange(st_kosev) 
        sosedi, min_vrednost = najdi_sosede_1d(kosi, izbrana_kosarica, razdalja) #iz nase pomozne funkcije dobimo sosede in minimalno vrednost
        kandidati = [k for k,v in sosedi.items() if v == min_vrednost] #poiscemo vse kandidate, ki so sosedi in imajo minimalno vrednost
        min_sosed = min(kandidati) #izmed kandidatov vzamemo tistega z najnizjim indeksom
        kosi[min_sosed] += 1 #minimalnemu sosedu dodamo zogo
    maksimum = max(kosi) #poiscemo maksimalno stevilo zog v posameznem kosu
    st_kosev_z_max_vrednostjo = kosi.count(max(kosi)) #prestejemo stevilo kosev, ki imajo maksimalno stevilo zog
    delez_kosev = st_kosev_z_max_vrednostjo / st_kosev #izracunamo delez kosev z maksimalnim stevilom zog
    konec = time.time()
    casovna_zahtevnost = f'{konec-zacetek}' #izracunamo casovno zahtevnost algoritma
    return maksimum, casovna_zahtevnost, delez_kosev
\end{verbatim}

\noindent V naslednjem koraku bova program prilagodila tako, da bova število sosedov in večkratnost žog vključila kot spremenljivki,
na koncu pa ga bova prevedla še na dve oz.\ tri dimenzije.

\section{Cilj}
Cilj najne seminarske naloge je preučevati vpliv različnih spremenljivk (število sosedov, večkratnost žog) na maksimalno število žog
v košari, ko so te razporejene v krogu, v torusu ali v treh dimezijah. Predstavljeni problem je problem teorije verjetnosti in pričakujeva,
da bova lahko maksimalno število žog, v odvisnosti od omenjenih parametrov, napovedala z verjetnostjo blizu 1.

\end{document}